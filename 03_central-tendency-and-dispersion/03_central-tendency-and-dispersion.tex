\documentclass[12pt, aspectratio=169]{beamer}

\input{../header}


\title{Central Tendency and Dispersion}
\author{Md. Aminul Islam Shazid}
\date{}


\begin{document}
    {
		\setbeamertemplate{footline}{}    % NO FOOTLINE FOR THESE TWO FRAMES
		\addtocounter{framenumber}{-2}    % not counting the title page and the outline in frame numbers

		\begin{frame}
			\titlepage
		\end{frame}

		\begin{frame}{Outline}
            \vfill
			\tableofcontents[subsectionstyle=hide]
            \vfill
		\end{frame}
	}

	\section{Introduction}

    \begin{frame}{Central Tendency}
        \begin{itemize}
            \item Observations of a variable tend to gather around a single value, this is known as central tendency
            \item Central tendency is a descriptive measure that represents the center or typical value of a variable
            \item It provides a summary of the values of the variable
        \end{itemize}
    \end{frame}


    \begin{frame}{Central Tendency (cont.)}
        \begin{itemize}
            \item Mean:
            \begin{itemize}
                \item Arithmetic mean
                \item Geometric mean
                \item Harmonic mean
            \end{itemize}
            \item Median
            \item Mode
        \end{itemize}
        These are different \textit{measures} of central tendency.\\[0.25em]
        Depending on the \text{shape} of the distribution and the presence of outliers, different measures are used.
    \end{frame}


    \begin{frame}{Characteristics of a Good Measure}
        \begin{itemize}
            \item Clear and unambiguous definition so that the same data provides the same value of the measure
            \item Easy to understand and calculate
            \item Based on all or most of the observations in the sample
            \item Not unduly affected by outliers so that a few outliers does not distort the result too much
            \item Representative of the distribution so that the value lies within the range of the data and and describe its central location
            \item Capable of further mathematical treatment so that it can be used for further analysis
        \end{itemize}
    \end{frame}


    \begin{frame}{When to Use Mean, Median or Mode}
        
    \end{frame}

    
    \section{Dispersion}


    \section{Outlier}


    \section{Boxplot}


    \section*{Questions?}


\end{document}