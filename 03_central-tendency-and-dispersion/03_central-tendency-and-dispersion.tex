\documentclass[12pt, aspectratio=169]{beamer}

\input{../header}
\usepackage{booktabs}


\title{Central Tendency and Dispersion}
\author{Md. Aminul Islam Shazid}
\date{}


\begin{document}
    {
		\setbeamertemplate{footline}{}    % NO FOOTLINE FOR THESE TWO FRAMES
		\addtocounter{framenumber}{-2}    % not counting the title page and the outline in frame numbers

		\begin{frame}
			\titlepage
		\end{frame}

		\begin{frame}{Outline}
            \vfill
			\tableofcontents[subsectionstyle=hide]
            \vfill
		\end{frame}
	}

	\section{Introduction}

    \begin{frame}{Central Tendency}
        \begin{itemize}
            \item Observations of a variable tend to gather around a single value, this is known as central tendency
            \item Central tendency is a descriptive measure that represents the center or typical value of a variable
            \item It provides a summary of the values of the variable
        \end{itemize}
    \end{frame}


    \begin{frame}{Central Tendency (cont.)}
        \begin{itemize}
            \item Mean:
            \begin{itemize}
                \item Arithmetic mean
                \item Geometric mean
                \item Harmonic mean
            \end{itemize}
            \item Median
            \item Mode
        \end{itemize}
        These are different \textit{measures} of central tendency. They represent the ``average" value of a dataset in different ways.\\[0.25em]
        Depending on the \text{shape} of the distribution and the presence of outliers, different measures are used.
    \end{frame}


    \begin{frame}{Characteristics of a Good Measure}
        \begin{itemize}
            \item Clear and unambiguous definition so that the same data provides the same value of the measure
            \item Easy to understand and calculate
            \item Based on all or most of the observations in the sample
            \item Not unduly affected by outliers so that a few outliers does not distort the result too much
            \item Representative of the distribution so that the value lies within the range of the data and and describe its central location
            \item Capable of further mathematical treatment so that it can be used for further analysis
        \end{itemize}
    \end{frame}


    \begin{frame}{Arithmetic Mean}
        \begin{itemize}
            \item The arithmetic mean is the sum of all observations divided by the number of observations
            \item For a some values $x_1, x_2, \dots, x_n$ of a variable $X$, the arithmetic mean is
            \[
                \bar{x} = \frac{1}{n}\sum_{i=1}^{n} x_i
            \]
            \item It uses all observations in the dataset
            \item The arithmetic mean is easy to compute and interpret
            \item It is \textit{sensitive} to extreme values (outliers)
            \item Therefore, it is most appropriate for numerical data that are symmetrically distributed
        \end{itemize}
    \end{frame}


    \begin{frame}{Example: Arithmetic Mean for Grouped Data}
        \begin{center}
            \begin{tabular}{lcc}
                \toprule
                \textbf{Value, $x_i$} & \textbf{Frequency}, $f_i$ & $f_i \cdot x_i$ \\
                \midrule
                55 & 7  & 385\\
                60 & 10 & 600\\
                62 & 6  & 372\\
                65 & 4  & 260\\
                67 & 3  & 201\\
                \bottomrule
                \textbf{Total:} & 30 & 1818
            \end{tabular}
        \end{center}
        The mean, $\bar{x} = \frac{\sum_{i=1}^{n} f_i x_i}{\sum_{i=1}^{n} f_i} = \frac{1818}{30}=60.6$
    \end{frame}


    \begin{frame}{Weighted Mean}
        \begin{itemize}
            \item When caluclating average, sometimes some values may be more important than other values
            \item In the previous example, the observations appeared different number of times
            \item Therefore, each value has different level of influence over the center of the distribution
            \item This is called the weight of each value
            \item Another example is the calculation of CGPA where the total credit of each semester is the weight of the corresponding GPA
        \end{itemize}
    \end{frame}


    \begin{frame}{Geometric Mean}
        
    \end{frame}


    \begin{frame}{Harmonic mean}
        
    \end{frame}


    \begin{frame}{Median}
        \begin{itemize}
            \item The median is the middle value of a dataset when the observations are arranged in ascending or descending order
            \item If the number of observations $n$ is odd, the median is the $\frac{n+1}{2}$-th observation
            \item If $n$ is even, the median is the average of the $\frac{n}{2}$-th and $\left(\frac{n}{2}+1\right)$-th observations
            \item The median divides the dataset into two equal halves
            \item Therefore, it is the value below which 50\% of the data lies
            \item It is not affected by extreme values (outliers)
            \item Therefore, it is useful for skewed distributions or data with outliers
        \end{itemize}
    \end{frame}


    \begin{frame}{Mode}
        
    \end{frame}


    \begin{frame}{Trimmed Mean}
        \begin{itemize}
            \item The trimmed mean is a measure of central tendency obtained by removing a fixed proportion of the smallest and largest observations.
            \item After trimming, the arithmetic mean is computed using the remaining data.
            \item A $p\%$ trimmed mean removes the lowest $p\%$ and highest $p\%$ of the data.
            \item It is less sensitive to extreme values than the arithmetic mean.
            \item The trimmed mean provides a balance between the mean and the median.
            \item It is useful when the data contain outliers or are moderately skewed.
        \end{itemize}
    \end{frame}


    \begin{frame}{When to Use Mean, Median or Mode}
        
    \end{frame}


    \begin{frame}{Quantile}
        
    \end{frame}


    \begin{frame}{Quartile}
        
    \end{frame}


    \begin{frame}{Percentile}
        
    \end{frame}

    
    \section{Dispersion}


    \begin{frame}{Range}
        
    \end{frame}


    \begin{frame}{Inter-quartile Range}
        
    \end{frame}


    \begin{frame}{Mean Deviation}
        
    \end{frame}


    \begin{frame}{Standard Deviation}
        
    \end{frame}


    \begin{frame}{Variance}
        
    \end{frame}


    \begin{frame}{Coefficient of Variation}
        
    \end{frame}


    \section{Outlier}


    \section{Boxplot}


    \section*{Questions?}


\end{document}