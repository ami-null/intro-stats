\documentclass[12pt, aspectratio=169]{beamer}

\input{../header}
\usepackage{booktabs}


\title{Central Tendency and Dispersion}
\author{Md. Aminul Islam Shazid}
\date{}


\begin{document}
    {
		\setbeamertemplate{footline}{}    % NO FOOTLINE FOR THESE TWO FRAMES
		\addtocounter{framenumber}{-2}    % not counting the title page and the outline in frame numbers

		\begin{frame}
			\titlepage
		\end{frame}

		\begin{frame}{Outline}
            \vfill
			\tableofcontents[subsectionstyle=hide]
            \vfill
		\end{frame}
	}

	\section{Introduction}

    \begin{frame}{Central Tendency}
        \begin{itemize}
            \item Observations of a variable tend to gather around a single value, this is known as central tendency
            \item Central tendency is a descriptive measure that represents the center or typical value of a variable
            \item It provides a summary of the values of the variable
        \end{itemize}
    \end{frame}


    \begin{frame}{Central Tendency (cont.)}
        \begin{itemize}
            \item Mean:
            \begin{itemize}
                \item Arithmetic mean
                \item Geometric mean
                \item Harmonic mean
            \end{itemize}
            \item Median
            \item Mode
        \end{itemize}
        These are different \textit{measures} of central tendency. They represent the ``average" value of a dataset in different ways.\\[0.25em]
        Depending on the \text{shape} of the distribution and the presence of outliers, different measures are used.
    \end{frame}


    \begin{frame}{Characteristics of a Good Measure}
        \begin{itemize}
            \item Clear and unambiguous definition so that the same data provides the same value of the measure
            \item Easy to understand and calculate
            \item Based on all or most of the observations in the sample
            \item Not unduly affected by outliers so that a few outliers does not distort the result too much
            \item Representative of the distribution so that the value lies within the range of the data and and describe its central location
            \item Capable of further mathematical treatment so that it can be used for further analysis
        \end{itemize}
    \end{frame}


    \begin{frame}{Arithmetic Mean}
        \begin{itemize}
            \item The arithmetic mean is the sum of all observations divided by the number of observations
            \item For a some values $x_1, x_2, \dots, x_n$ of a variable $X$, the arithmetic mean is
            \[
                \bar{x} = \frac{1}{n}\sum_{i=1}^{n} x_i
            \]
            \item It uses all observations in the dataset
            \item The arithmetic mean is easy to compute and interpret
            \item It is \textit{sensitive} to extreme values (outliers)
            \item Therefore, it is most appropriate for numerical data that are symmetrically distributed
        \end{itemize}
    \end{frame}


    \begin{frame}{Example: Arithmetic Mean From Frequency Table}
        \begin{center}
            \begin{tabular}{lcc}
                \toprule
                \textbf{Value, $x_i$} & \textbf{Frequency}, $f_i$ & $f_i \cdot x_i$ \\
                \midrule
                55 & 7  & 385\\
                60 & 10 & 600\\
                62 & 6  & 372\\
                65 & 4  & 260\\
                67 & 3  & 201\\
                \bottomrule
                \textbf{Total:} & 30 & 1818
            \end{tabular}
        \end{center}
        The mean, $\bar{x} = \frac{\sum_{i=1}^{n} f_i x_i}{\sum_{i=1}^{n} f_i} = \frac{1818}{30}=60.6$.\\[0.5em]

        If the data is grouped, then the class midpoints are treated as $x_i$.
    \end{frame}


    \begin{frame}{Arithmetic Mean for Grouped Data}
        \begin{center}
            \begin{tabular}{lccc}
                \toprule
                \textbf{Category} & \textbf{Class midpoint}, $x_i$ & \textbf{Frequency}, $x_i$ & $f_i \cdot x_i$\\
                \midrule
                5-30    & 17.5  & 7  & 122.5 \\
                30-55   & 42.5  & 10 & 425 \\
                55-80   & 67.5  & 6  & 405 \\
                80-105  & 92.5  & 4  & 370 \\
                105-130 & 117.5 & 3  & 352.5 \\
                \bottomrule
                 & \textbf{Total}: & 30 & 1675
            \end{tabular}
        \end{center}
        Mean = $1675/30 = 55.83$
    \end{frame}


    \begin{frame}{Weighted Mean}
        \begin{itemize}
            \item When caluclating average, sometimes some values may be more important than other values
            \item In the previous example, the observations appeared different number of times
            \item Therefore, each value has different level of influence over the center of the distribution
            \item This is called the weight of each value
            \item Another example is the calculation of CGPA where the total credit of each semester is the weight of the corresponding GPA
        \end{itemize}
    \end{frame}


    \begin{frame}{Geometric Mean}
        \begin{itemize}
            \item The geometric mean is a measure of central tendency defined as the $n$-th root of the product of $n$ positive observations
            \item For positive data $x_1, x_2, \dots, x_n$, the geometric mean is
            \[
                G = \left( \prod_{i=1}^{n} x_i \right)^{1/n}
            \]
            \item It is only defined for positive values
            \item The geometric mean is appropriate for data involving ratios, rates, or growth factors
            \item It reduces the influence of very large values compared to the arithmetic mean
            \item The geometric mean is commonly used for percentage changes and financial returns
        \end{itemize}
    \end{frame}


    \begin{frame}{Geometric Mean for Grouped Data}
        \begin{itemize}
            \item For grouped data, the geometric mean is calculated using class frequencies
            \item Let $x_1, x_2, \dots, x_k$ be the class midpoints and $f_1, f_2, \dots, f_k$ the corresponding frequencies
            \item The geometric mean is given by
            \[
                G = \left( \prod_{i=1}^{k} x_i^{\,f_i} \right)^{1/n},
            \]
            where $n = \sum_{i=1}^{k} f_i$
            \item In practice, the computation is often simplified using logarithms:
            \[
                \log G = \frac{1}{n} \sum_{i=1}^{k} f_i \log x_i
            \]
        \end{itemize}
    \end{frame}


    \begin{frame}{Harmonic mean}
        \begin{itemize}
            \item The harmonic mean is a measure of central tendency defined as the reciprocal of the arithmetic mean of reciprocals
            \item For positive data $x_1, x_2, \dots, x_n$, the harmonic mean is
            \[
                H = \frac{n}{\sum_{i=1}^{n} \frac{1}{x_i}}
            \]
            \item It is only defined for positive values
            \item The harmonic mean gives more weight to smaller observations
            \item It is appropriate for averaging rates or ratios, such as speeds or densities
            \item The harmonic mean is strongly affected by very small values
        \end{itemize}
    \end{frame}


    \begin{frame}{Harmonic Mean for Grouped Data}
        \begin{itemize}
            \item For grouped data, the harmonic mean is calculated using class frequencies
            \item Let $x_1, x_2, \dots, x_k$ be the class midpoints and $f_1, f_2, \dots, f_k$ the corresponding frequencies.
            \item The harmonic mean is given by
            \[
                H = \frac{n}{\sum_{i=1}^{k} \dfrac{f_i}{x_i}},
            \]
            where $n = \sum_{i=1}^{k} f_i$
        \end{itemize}
    \end{frame}


    \begin{frame}{Mode}
        \begin{itemize}
            \item The mode is the value that occurs most frequently in a dataset
            \item A dataset may have:
            \begin{itemize}
                \item one mode (unimodal),
                \item two modes (bimodal), or
                \item more than two modes (multimodal)
            \end{itemize}
            \item The mode can be used for both numerical and categorical data
            \item A dataset may have no mode if all values occur with the same frequency
            \item The mode is not affected by extreme values
            \item For grouped data, the mode is estimated using the modal class
        \end{itemize}
    \end{frame}


    \begin{frame}{Mode for Grouped Data}
        \[
            \textsf{Mode} = L_0 + \frac{l_1}{l_1 + l_2} \times c,
        \]
        where:
        \begin{itemize}
            \item $L_0$ is the lower limit of the modal class (class with the highest frequency)
            \item $l_1$ is the difference in Frequency between the modal class and the pre-modal class
            \item $l_2$ is the difference in Frequency between the modal class and the post-modal class
            \item c is the class interval
        \end{itemize}
    \end{frame}


    \begin{frame}{Example: Mode for Grouped Data}
        \begin{center}
            \begin{tabular}{lcc}
                \toprule
                \textbf{Group} & \textbf{Frequency}\\
                \midrule
                5-30    & 7 \\
                30-55   & 10 \\
                55-80   & 6 \\
                80-105  & 4 \\
                105-130 & 3 \\
                \bottomrule
            \end{tabular}
        \end{center}
        Mode = $30 + \frac{3}{3+4} \times 30 = 40.71$
    \end{frame}


    \begin{frame}{Median}
        \begin{itemize}
            \item The median is the middle value of a dataset when the observations are arranged in ascending or descending order
            \item If the number of observations $n$ is odd, the median is the $\frac{n+1}{2}$-th observation
            \item If $n$ is even, the median is the average of the $\frac{n}{2}$-th and $\left(\frac{n}{2}+1\right)$-th observations
            \item The median divides the dataset into two equal halves
            \item Therefore, it is the value below which 50\% of the data lies
            \item It is not affected by extreme values (outliers)
            \item Therefore, it is useful for skewed distributions or data with outliers
        \end{itemize}
    \end{frame}


    \begin{frame}{Median for Grouped Data}
        \[
            \textsf{Median} = L_m + \frac{\frac{n}{2} - F_c}{f_m} \times c,
        \]
        where:
        \begin{itemize}
            \item $L_m$ = lower limit of the median group, it is the group in which relative cumulative frequency is equal to 0.5 (50\%) or the first group in which relative cumulative frequency exceeds 0.5
            \item n = sample size
            \item $F_c$ = cumulative frequency of the pre-median class
            \item $f_m$ = frequency of the median class
        \end{itemize}
    \end{frame}


    \begin{frame}{Example: Median for Grouped Data}
        \begin{center}
            \begin{tabular}{lcc}
                \toprule
                \textbf{Group} & \textbf{Frequency} & \textbf{Cumulative Frequency}\\
                \midrule
                5-30    & 7  & 7\\
                30-55   & 10 & 17\\
                55-80   & 6  & 23\\
                80-105  & 4  & 27\\
                105-130 & 3  & 30\\
                \bottomrule
            \end{tabular}
        \end{center}
        Here, sample size is 30. Since 50\% of 30 is 15, the second group is the median class.\\[0.25em]

        Median = $L_m + \frac{\frac{n}{2} - F_c}{f_m} \times c = 30 + \frac{\frac{30}{2} - 7}{10} \times 25 = 50$.
    \end{frame}


    \begin{frame}{Trimmed Mean}
        \begin{itemize}
            \item The trimmed mean is a measure of central tendency obtained by removing a fixed proportion of the smallest and largest observations
            \item After trimming, the arithmetic mean is computed using the remaining data
            \item A $p\%$ trimmed mean removes the lowest $p\%$ and highest $p\%$ of the data
            \item It is less sensitive to extreme values than the arithmetic mean
            \item The trimmed mean provides a balance between the mean and the median
            \item It is useful when the data contain outliers or are moderately skewed
        \end{itemize}
    \end{frame}


    \begin{frame}{Quantile}
        \begin{itemize}
            \item Quantiles are values that divide an ordered dataset into equal parts
            \item Each part contains the same proportion of observations
            \item Common quantiles include:
            \begin{itemize}
                \item Quartiles: divide the data into four equal parts
                \item Deciles: divide the data into ten equal parts
                \item Percentiles: divide the data into one hundred equal parts
            \end{itemize}
            \item The median is the second quartile ($Q_2$) or the 50th percentile, it divides the data in two parts
            \item Quantiles are useful for describing the distribution and spread of data
        \end{itemize}
    \end{frame}


    \begin{frame}{Quartile}
        \begin{itemize}
            \item Quartiles are values that divide an ordered dataset into four equal parts
            \item Each part contains approximately 25\% of the observations
            \item The three quartiles are:
            \begin{itemize}
                \item First quartile ($Q_1$): 25th percentile
                \item Second quartile ($Q_2$): 50th percentile (the median)
                \item Third quartile ($Q_3$): 75th percentile
            \end{itemize}
            \item Quartiles are used to describe the spread and position of data
            \item Sometimes, the minimum value is referred to as the 0th quartile and the maximum value as the 4th quartile
        \end{itemize}
    \end{frame}


    \begin{frame}{Percentile}
        \begin{itemize}
            \item Percentiles divide an ordered dataset into 100 equal parts
            \item Each percentile represents 1\% of the observations
            \item The $p$-th percentile is the value below which $p\%$ of the data lie
            \item The median is the 50th percentile
            \item Percentiles are widely used in examinations, test scores, and rankings
        \end{itemize}
        Someone's IQ score being 90th percentile means that the score is above 90\% of the population.
    \end{frame}


    \begin{frame}{Decile}
        \begin{itemize}
            \item Deciles divide an ordered dataset into 10 equal parts
            \item Each decile represents 10\% of the observations
            \item The $k$-th decile ($D_k$) is the value below which $k \times 10\%$ of the data lie
            \item The fifth decile ($D_5$) coincides with the median
            \item Deciles are useful for studying the distribution of data in broader groups
        \end{itemize}
    \end{frame}

    
    \section{Dispersion}


    \begin{frame}{Range}
        
    \end{frame}


    \begin{frame}{Inter-quartile Range}
        Used when the range is unduly affected by outliers
    \end{frame}


    \begin{frame}{Mean Deviation}
        
    \end{frame}


    \begin{frame}{Mean Absolute Deviation}
        
    \end{frame}


    \begin{frame}{Standard Deviation}
        
    \end{frame}


    \begin{frame}{Variance}
        
    \end{frame}


    \begin{frame}{Coefficient of Variation}
        
    \end{frame}


    \section{Outlier}


    \section{Boxplot}


    \section*{Questions?}


\end{document}