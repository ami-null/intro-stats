\documentclass[12pt, aspectratio=169]{beamer}

\input{../header}


\title{Introduction to Statistics}
\author{Md. Aminul Islam Shazid}
\date{}


\begin{document}
    {
		\setbeamertemplate{footline}{}    % NO FOOTLINE FOR THESE TWO FRAMES
		\addtocounter{framenumber}{-2}    % not counting the title page and the outline in frame numbers

		\begin{frame}
			\titlepage
		\end{frame}

		\begin{frame}{Outline}
            \vfill
			\tableofcontents[subsectionstyle=hide]
            \vfill
		\end{frame}
	}

	\section{Introduction}

	\begin{frame}{What is Statistics?}
		\begin{itemize}
			\item Collecting data
			\item Gaining insights from data
			\item Making decisions based on the insights gained from the data
		\end{itemize}
	\end{frame}


	\begin{frame}{Definition}
		Statistics can be defined as the art and science of:
		\begin{itemize}
			\item collecting, cleaning and organizing data
			\item summarizing and analyzing data
			\item presenting the summary or the analysis
			\item interpreting the analysis results
			\item gaining insights through analysis of data
			\item and finally, drawing valid conclusions and making sound decisions through the use of data.
		\end{itemize}
	\end{frame}


	\begin{frame}{Why Statistics is Necessary}
		\begin{itemize}
			\item 
		\end{itemize}
	\end{frame}


	\begin{frame}{Popuplation and Sample}
		\begin{itemize}
			\item \textbf{Population} is the collection/set of all items or individuals of interest in a given study
			\item \textbf{Sample} is a \textit{representative} portion of the population
		\end{itemize}

		For example:
		\begin{itemize}
			\item A study may target all the people in Bangladesh. However, it is unfeasible to collect information of everyone in the country in a timely or cost-effective effective way
			\item Therefore, data is collected from only a small portion of people from \textit{all over the country}, this is called sampling. The individuals in a sample are usually selected randomly
		\end{itemize}
  	\end{frame}


	\begin{frame}{Census and Survey}
		
	\end{frame}


	\begin{frame}{Parameter and Statistic}
		\begin{itemize}
			\item A \textbf{parameter} is a characteristic or function of every objects or individuals in a population. For a fixed population, it is a fixed (but, usually unknown) value
			\item A \textbf{statistic} is a characteristic or function of every objects or individuals in a sample.
			\item A \textbf{statistic} is used to \textit{estimate} a \textbf{parameter}
		\end{itemize}
	\end{frame}


	\begin{frame}{Parameter and Statistic (cont.)}
		\begin{itemize}
			\item For a fixed population, the value of a parameter is fixed (but usually unknown)
			\item However, due to randomization, different samples can include different individuals from a population
			\item Therefore, the value of a statistic can vary across different samples
		\end{itemize}
	\end{frame}


	\begin{frame}{Parameter and Statistic (cont.)}
		For example:
		\begin{itemize}
			\item Suppose the goal is to find the average height of the students of a class
			\item The population average is a fixed value and it is unknown unless data is collected from everyone in the class
			\item If the heights of a some students are collected as a random sample, then we can estimate the population average using the sample average
			\item If another sample is collcted, the same individuals as the first sample may not be selected, therefore, the estimate shall be different from the first estimate
		\end{itemize}
	\end{frame}


	\begin{frame}{Types of Statistics}
		\begin{itemize}
			\item \textbf{Descriptive statistics:} Methods for organizing, summarizing and presenting data in an informative way. For example:
			\begin{itemize}
				\item A hypothetical customer survey finds that 50\% of the customers are satisfied with a product
			\end{itemize}
			\item \textbf{Inferential statistic:} Methods for using sample data to make predictions, test hypotheses, and generalize conclusions about a larger population. For example:
			\begin{itemize}
				\item A study finds association between smoking and cancer
			\end{itemize}
		\end{itemize}
		
	\end{frame}


\end{document}