\documentclass[12pt, aspectratio=169]{beamer}

\input{../header}


\title{Introduction to Statistics}
\author{Md. Aminul Islam Shazid}
\date{}


\begin{document}
    {
		\setbeamertemplate{footline}{}    % NO FOOTLINE FOR THESE TWO FRAMES
		\addtocounter{framenumber}{-2}    % not counting the title page and the outline in frame numbers

		\begin{frame}
			\titlepage
		\end{frame}

		\begin{frame}{Outline}
            \vfill
			\tableofcontents[subsectionstyle=hide]
            \vfill
		\end{frame}
	}

	\section{Introduction}

	\begin{frame}{What is Statistics?}
		\begin{itemize}
			\item Collecting data
			\item Gaining insights from data
			\item Making decisions based on the insights gained from the data
		\end{itemize}
	\end{frame}


	\begin{frame}{Definition}
		Statistics can be defined as the art and science of:
		\begin{itemize}
			\item collecting, cleaning and organizing data
			\item summarizing and analyzing data
			\item presenting the summary or the analysis
			\item interpreting the analysis results
			\item gaining insights through analysis of data
			\item and finally, drawing valid conclusions and making sound decisions through the use of data.
		\end{itemize}
	\end{frame}


	\begin{frame}{Why Statistics is Necessary}
		\begin{itemize}
			\item 
		\end{itemize}
	\end{frame}


	\section{Some Basic Statistical Concepts}


	\begin{frame}{Popuplation and Sample}
		\begin{itemize}
			\item \textbf{Population} is the collection/set of all items or individuals of interest in a given study
			\item \textbf{Sample} is a \textit{representative} portion of the population
		\end{itemize}

		For example:
		\begin{itemize}
			\item A study may target all the people in Bangladesh. However, it is unfeasible to collect information of everyone in the country in a timely or cost-effective effective way
			\item Therefore, data is collected from only a small portion of people from \textit{all over the country}, this is called sampling. The individuals in a sample are usually selected randomly
		\end{itemize}
  	\end{frame}


	\begin{frame}{Census and Survey}
		
	\end{frame}


	\begin{frame}{Parameter and Statistic}
		\begin{itemize}
			\item A \textbf{parameter} is a characteristic or function of every objects or individuals in a population. For a fixed population, it is a fixed (but, usually unknown) value
			\item A \textbf{statistic} is a characteristic or function of every objects or individuals in a sample.
			\item A \textbf{statistic} is used to \textit{estimate} a \textbf{parameter}
		\end{itemize}
	\end{frame}


	\begin{frame}{Parameter and Statistic (cont.)}
		\begin{itemize}
			\item For a fixed population, the value of a parameter is fixed (but usually unknown)
			\item However, due to randomization, different samples can include different individuals from a population
			\item Therefore, the value of a statistic can vary across different samples
		\end{itemize}
	\end{frame}


	\begin{frame}{Parameter and Statistic (cont.)}
		For example:
		\begin{itemize}
			\item Suppose the goal is to find the average height of the students of a class
			\item The population average is a fixed value and it is unknown unless data is collected from everyone in the class
			\item If the heights of a some students are collected as a random sample, then we can estimate the population average using the sample average
			\item If another sample is collcted, the same individuals as the first sample may not be selected, therefore, the estimate shall be different from the first estimate
		\end{itemize}
	\end{frame}


	\begin{frame}{Types of Statistics}
		\begin{itemize}
			\item \textbf{Descriptive statistics:} Methods for organizing, summarizing and presenting data in an informative way. For example:
			\begin{itemize}
				\item A hypothetical customer survey finds that 50\% of the customers are satisfied with a product
			\end{itemize}
			\item \textbf{Inferential statistic:} Methods for using sample data to make predictions, test hypotheses, and generalize conclusions about a larger population. For example:
			\begin{itemize}
				\item A study finds association between smoking and cancer
			\end{itemize}
		\end{itemize}
	\end{frame}


	\section{Variable and Measurement}


	\begin{frame}{Variable}
		\begin{itemize}
			\item Variable means something that can vary
			\item It is a characteristic that can vary across individuals or objects or items or cases of a phenomenon
			\item For example:
			\begin{itemize}
				\item Age
				\item Gender
				\item Socioeconomic status
				\item Temperature
			\end{itemize}
		\end{itemize}
	\end{frame}


	\begin{frame}{Types of Variables}
		Variables:
		\begin{itemize}
			\item Qualitative
			\item Quantitative:
			\begin{itemize}
				\item Discreet
				\item Continuous
			\end{itemize}
		\end{itemize}
	\end{frame}


	\begin{frame}{Qualitative vs Qualitative}
		\begin{itemize}
			\item \textbf{Qualitative:} Qualitative variables describe qualities and are categorical. These are non-numerical and descriptive values that represent attributes or categories. For example:
			\begin{itemize}
				\item Name of a person
				\item Gender
				\item Hair colour
			\end{itemize}
			\item \textbf{Quantitative:} Quantitative variables measure quantities with numbers. These are numeric data that can be counted or measured, allowing for mathematical calculations. For example:
			\begin{itemize}
				\item Height 
				\item Temperature
				\item Number of students in a class 
			\end{itemize}
		\end{itemize}
	\end{frame}


	\begin{frame}{Discreet vs Continuous}
		\begin{itemize}
			\item \textbf{Discreet:} \textit{Countable}, usually whole numbers. Finite number of possible values. For example:
			\begin{itemize}
				\item Number of students in a class
			\end{itemize}
			\item \textbf{Continuous:} \textit{Measurable}, can have fractional values. Can take values in a given range. Infinite number of possible values in any range. For example:
			\begin{itemize}
				\item Height
			\end{itemize}
		\end{itemize}
	\end{frame}


	\begin{frame}{Scales of Variables or Level of Measurement}
		Scales of variables, or levels of measurement, define how data is categorized and quantified/measured. It dictates which statistical analysis is appropriate for a variable.\\[1em]
		There are four levels of measurement:
		\begin{itemize}
			\item Nominal
			\item Ordinal
			\item Interval
			\item Ratio
		\end{itemize}
		\vspace{1em}
		Nominal and ordinal are for categorical or qualitative data, and the last two are for quantitative data.
	\end{frame}


	\begin{frame}{Nominal}
		\begin{itemize}
			\item Nominal scale is for variables with categories with no inherent order or ranking (labels/names only)
			\item Example:
			\begin{itemize}
				\item Gender
				\item Religion
			\end{itemize}
			\item Allowed operations:
			\begin{itemize}
				\item We can only \textbf{count} the occurence or frequency of each of the categories of a nominal variable
			\end{itemize}
		\end{itemize}
	\end{frame}


	\begin{frame}{Ordinal}
		\begin{itemize}
			\item Ordinal scale is for variables with categories that can be \textit{ordered} or ranked, but differences between ranks are not necessarily equal or measurable
			\item Example:
			\begin{itemize}
				\item Satisfaction level (low, medium, high)
				\item Race position (first, second, third)
				\item Socioeconomic status (lower, mid, upper)
			\end{itemize}
			\item Allowed operations:
			\begin{itemize}
				\item We can \textbf{count} the frequency of each of the categories of an ordianl variable as well as \textbf{order} the categories
			\end{itemize}
		\end{itemize}
	\end{frame}


	\begin{frame}{Interval}
		\begin{itemize}
			\item It is for numerical variables. The values are ordered, with equal, meaningful \textit{intervals} between values
			\item There is no true zero point
			\item Example:
			\begin{itemize}
				\item Temperature (0°C does not imply the lack of heat or thermal energy)
				\item Years on a Calendar
			\end{itemize}
			\item Since there is no true zero point, taking ratio of two values of an interval scale variable is meaningless
			\item Allowed operations:
			\begin{itemize}
				\item We can \textbf{add}, \textbf{subtract} different values of an interval scale variable
			\end{itemize}
		\end{itemize}
	\end{frame}


	\begin{frame}{Ratio}
		\begin{itemize}
			\item An ratio scale variable includes all properties of interval scales, plus a true zero point, allowing for meaningful ratios
			\item Example:
			\begin{itemize}
				\item Height
				\item Weight
				\item Income
			\end{itemize}
			\item Allowed operations:
			\begin{itemize}
				\item All arithmetic operations (addition, subtraction, multiplication, division, ratios)
			\end{itemize}
		\end{itemize}
	\end{frame}


	\section{Sources of Data}


	\begin{frame}{Data}
		\begin{itemize}
			\item Data in statistics consists of numerical or qualitative facts collected for analysis
			\item They are recorded observations about individuals, objects, or events
			\item Data arise from measurement, counting, surveys, or experiments
			\item Each observation consists of one or more variables
			\item Usually presented in a tabular form, can also be in other forms (such as images, autio, video etc.)
			\item Statistics uses data to summarize information and draw conclusions about a larger context
		\end{itemize}
	\end{frame}

	
	\begin{frame}{Sources of Data}
		\begin{itemize}
			\item \textbf{Primary data}: data collected firsthand by the researcher for a specific study
			\item \textbf{Secondary data}: data previously collected by others for a different or a more general purpose
			\item \textbf{Tertiary data}: data that summarize, compile, or index primary and secondary sources
		\end{itemize}
	\end{frame}


	\begin{frame}{Primary Data}
		Data collected firsthand by the researcher for a specific study.
		\begin{itemize}
			\item Advantages:
			\begin{itemize}
				\item Directly relevant to the research objectives
				\item Greater control over data quality and measurement methods
				\item Up-to-date and specific to the population of interest
				\item Clear understanding of how the data were collected
			\end{itemize}
			\item Disadvantages:
			\begin{itemize}
				\item Time-consuming to collect
				\item Often expensive in terms of money and resources
				\item Requires careful planning and technical expertise
				\item Limited scope compared to large existing datasets
			\end{itemize}
		\end{itemize}
	\end{frame}


	\begin{frame}{Secondary Data}
		Data previously collected by others for a different or a more general purpose
		\begin{itemize}
			\item Advantages:
			\begin{itemize}
				\item Quick and inexpensive to obtain
				\item Often covers large populations or long time periods
				\item Useful for comparisons and trend analysis
				\item No need for data collection infrastructure
			\end{itemize}
			\item Disadvantages:
			\begin{itemize}
				\item May not exactly match the research objectives
				\item Limited control over data quality and measurement methods
				\item Possible issues with outdated or incomplete data
				\item Documentation and variable definitions may be unclear
			\end{itemize}
		\end{itemize}
	\end{frame}


	\begin{frame}{Other Ways to Classify Data}
		\begin{itemize}
			\item Observational data
			\item Experimental data
			\item Administrative records/registers
			\item Publications and reports
		\end{itemize}
	\end{frame}


	\section*{Questions?}


\end{document}