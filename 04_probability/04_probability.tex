\documentclass[12pt, aspectratio=169]{beamer}

\input{../header}
\usepackage{booktabs}


\title{Probability}
\author{Md. Aminul Islam Shazid}
\date{}


\begin{document}
    {
		\setbeamertemplate{footline}{}    % NO FOOTLINE FOR THESE TWO FRAMES
		\addtocounter{framenumber}{-2}    % not counting the title page and the outline in frame numbers

		\begin{frame}
			\titlepage
		\end{frame}

		\begin{frame}{Outline}
            \vfill
			\tableofcontents[subsectionstyle=hide]
            \vfill
		\end{frame}
	}

	\section{Introduction}

    \begin{frame}{What is Probability?}
        Probability deals with uncertainty and quantifies how likely an event is to occur.
		\begin{itemize}
			\item Many real-life situations involve uncertainty rather than certainty
			\item Probability helps us make informed decisions under uncertainty
			\item It provides a numerical measure of chance, between 0 and 1
			\item A probability close to 0 indicates a rare event
			\item A probability close to 1 indicates a highly likely event
		\end{itemize}
    \end{frame}


	\begin{frame}{Examples}
		Probability concepts appear naturally in daily activities.
		\begin{itemize}
			\item Weather forecasting: chance of rain tomorrow
			\item Medical testing: likelihood that a test result is correct
			\item Games and sports: chances of winning or losing
			\item Traffic planning: probability of congestion at a given time
			\item Finance: risk assessment and expected returns
		\end{itemize}
	\end{frame}


	\begin{frame}{Example: Tossing a Coin}
		\begin{itemize}
			\item The experiment consists of tossing a fair coin once
			\item Possible outcomes are Head (H) and Tail (T)
			\item Each outcome has an equal chance of occurring
			\item Probability of Head = $0.5$
			\item Probability of Tail = $0.5$
		\end{itemize}
	\end{frame}


	\section{Key Concept and Terms}

	\begin{frame}{Basic Principal of Counting}
		\begin{itemize}
			\item If an event can occur in $m$ possible ways and for each of the $m$ possible ways that the first event can occur, there are $n$ possible ways that a second event can occur, then there are in total $m \times n$ possible ways that the two events can occur together
			\item For example, if a person can go from place A to place B in three possible ways, and B to C in two ways, then there are a total of six ways to go from A to C
		\end{itemize}
	\end{frame}


	\begin{frame}{Generalized Basic Principle of Counting}
		\begin{itemize}
			\item If an event can occur in $𝑚_1$ possible ways and for each of the possible ways that the first event can occur, there are $𝑚_2$ possible ways that a second event can occur, and again for each of the $𝑚_1 \times 𝑚_2$ possible ways that the first two events can occur, there are $𝑚_3$ possible ways that a third event can occur, and so on, then there are in total $m_1 \times m_2 \times m_3...$ possible ways that all these events can occur together
		\end{itemize}
	\end{frame}

	\begin{frame}{Permutation}
		A permutation is an arrangement of objects where the order matters.
		\begin{itemize}
			\item Number of permutations of $r$ objects chosen from $n$ distinct objects:
			\[
				^nP_r = \frac{n!}{(n-r)!}
			\]
			\item Used when positions or order are important
			\item Example:
			\begin{itemize}
				\item Number of ways to arrange 3 students out of 5 in a row:
				\[
					^5P_3 = \frac{5!}{2!} = 60
				\]
			\end{itemize}
		\end{itemize}
	\end{frame}


	\begin{frame}{Combination}
	A combination is a selection of objects where the order does not matter.
		\begin{itemize}
			\item Number of combinations of $r$ objects chosen from $n$ distinct objects:
			\[
				^nC_r = \frac{n!}{r!(n-r)!}
			\]
			\item Used when only selection matters, not arrangement
			\item Example:
			\begin{itemize}
				\item Number of ways to choose 3 students from 5:
				\[
					^5C_3 = \frac{5!}{3!2!} = 10
				\]
			\end{itemize}
		\end{itemize}
	\end{frame}


	\begin{frame}{Experiment}
		\begin{itemize}
			\item An experiment is any process that can be repeated under certain conditions and that produces an observable result
			\item The result of an experiment is called an outcome
			\item Example:
			\begin{itemize}
				\item Tossing a coin or a dice
				\item Measuring daily rainfall
				\item Conducting chemical reactions
			\end{itemize}
		\end{itemize}		
	\end{frame}


	\begin{frame}{Outcome}
		\begin{itemize}
			\item An outcome is a single possible result of an experiment
			\item There can be one or more \emph{potential} outcomes
			\item Each experiment produces exactly one outcome
			\item Outcomes may be numerical or categorical
			\item Example:
			\begin{itemize}
				\item Getting a head when tossing a coin
				\item Getting a 4 when throwing a dice
			\end{itemize}
		\end{itemize}		
	\end{frame}


	\begin{frame}{Types of Experiment}
		Experiments can be categorized in to two types based on the nature of their outcome(s):
		\begin{itemize}
			\item Deterministic: outcome is known or can be predicted with certainty
			\item Random: outcome is unknown and cannot be predicted with certainty
		\end{itemize}
	\end{frame}


	\begin{frame}{Random Experiment}
	A random experiment is an experiment whose outcome cannot be predicted with certainty.
		\begin{itemize}
			\item The same experiment may produce different outcomes on repetition
			\item Potential outcomes are known, but which one will occur is uncertain
			\item Examples:
			\begin{itemize}
				\item Tossing a coin
				\item Rolling a dice
				\item Drawing a card from a shuffled deck
			\end{itemize}
		\end{itemize}
	\end{frame}


	\begin{frame}{Deterministic Experiment}
		A deterministic experiment is an experiment whose outcome can be predicted with certainty.
		\begin{itemize}
			\item Repeating the experiment under identical conditions gives the same result
			\item No randomness is involved
			\item Examples:
			\begin{itemize}
				\item Calculating the sum of two fixed numbers
				\item Measuring the boiling point of pure water at standard pressure
			\end{itemize}
		\end{itemize}
	\end{frame}


	\begin{frame}{Iteration (Trial or Repetition)}
	An iteration refers to repeating an experiment under identical conditions.
		\begin{itemize}
			\item Each repetition is called a trial
			\item Iterations help study long-run behavior of outcomes
			\item Examples:
			\begin{itemize}
				\item Tossing a coin 100 times
				\item Rolling a die repeatedly and recording outcomes
			\end{itemize}
		\end{itemize}
	\end{frame}


	\begin{frame}{Sample Space}
		The sample space is the set of all possible outcomes of a random experiment.
		\begin{itemize}
			\item Denoted by $S$.
			\item Each outcome is called a sample point
			\item Example:
			\begin{itemize}
				\item Tossing a coin once: \(S = \{H, T\}\)
				\item Rolling a die: \(S = \{1,2,3,4,5,6\}\)
				\item Tossing a coin twice: \(S = \{HH, HT, TH, TT\}\)
			\end{itemize}
		\end{itemize}
	\end{frame}


	\begin{frame}{Event}
		An event is any \emph{subset} of the sample space.
		\begin{itemize}
			\item An event may contain one or more outcomes
			\item A \emph{simple (elementary) event} contains exactly one outcome
			\item Example (dice roll):
			\begin{itemize}
				\item Event: 
				\begin{itemize}
					\item Getting an even number: $\{2,4,6\}$
					\item Getting four or higher: $\{4,5,6\}$
				\end{itemize}
				\item Simple event: getting a $4$: $\{4\}$
			\end{itemize}
		\end{itemize}
	\end{frame}


	\begin{frame}{Mutually Exclusive Events}
		Two or more events are mutually exclusive if they cannot occur simultaneously.
		\begin{itemize}
			\item They have no common outcomes
			\item For events $A$ and $B$:
			\[
				A \cap B = \varnothing
			\]
			\item Example (dice roll):
			\begin{itemize}
				\item $A$: getting an even number
				\item $B$: getting an odd number
			\end{itemize}
		\end{itemize}
	\end{frame}


	\begin{frame}{Collectively Exhaustive Events}
		Events are collectively exhaustive if their union covers the entire sample space.
		\begin{itemize}
			\item At least one of the events must occur
			\item For events $A_1, A_2, \dots, A_n$:
			\[
				A_1 \cup A_2 \cup \cdots \cup A_n = S
			\]
			\item Example:
			\begin{itemize}
				\item Tossing a coin: $A_1 = \{H\}$ and $A_2 = \{T\}$
			\end{itemize}
		\end{itemize}
	\end{frame}


	\begin{frame}{Impossible and Sure Events}
		\begin{itemize}
			\item An impossible event is an event that cannot occur
			\begin{itemize}
				\item Probability is 0
				\item Example: getting a 7 on a fair die
			\end{itemize}
			\item A sure (certain) event is an event that always occurs
			\begin{itemize}
				\item Probability is 1
				\item Example: getting a number less than 7 on a fair die
			\end{itemize}
		\end{itemize}
	\end{frame}


	\begin{frame}{Equally Likely Events}
		Events are equally likely if each has the same chance of occurring.
		\begin{itemize}
			\item Common in experiments with symmetry
			\item Example:
			\begin{itemize}
				\item Tossing a fair coin: $P(H) = P(T) = 0.5$
				\item Rolling a fair die: each outcome has probability $1/6$
			\end{itemize}
		\end{itemize}
	\end{frame}




	\section*{Thank you.}

\end{document}