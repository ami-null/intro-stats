\documentclass[12pt, aspectratio=169]{beamer}

\input{../header}
\usepackage{booktabs}
\usepackage{tikz}
\usetikzlibrary{patterns}


\title{Random Variable and Probabiilty Distribution}
\author{Md. Aminul Islam Shazid}
\date{}


\begin{document}
    {
		\setbeamertemplate{footline}{}    % NO FOOTLINE FOR THESE TWO FRAMES
		\addtocounter{framenumber}{-2}    % not counting the title page and the outline in frame numbers

		\begin{frame}
			\titlepage
		\end{frame}

		\begin{frame}{Outline}
            \vfill
			\tableofcontents[subsectionstyle=hide]
            \vfill
		\end{frame}
	}

	\section{Random Variable}

    \begin{frame}{Introduction}
        A variable which takes numerical values resulting from random experiments, is called a random variable (RV).

		\begin{itemize}
			\item There is a probability associated with each possible values
			\item Random variables are denoted by capital letters such as $X, Y, Z$ etc.
			\item Possible values are denoted by small letters such as $x, y, z$ etc.
			\item Example:
			\begin{itemize}
				\item Height of students
				\item Number of heads when tossing a coin three times
			\end{itemize}
		\end{itemize}
    \end{frame}


	\begin{frame}{Types of Random Variable}
		\begin{itemize}
			\item \textbf{Discrete random variable}: A random variable defined over a discrete sample space
			\begin{itemize}
				\item Number of students in a class, sample space, $S = {0, 1, 2, ..., \infty}$
				\item Number of correct answers in among $50$ questions, $S = {0, 1, 2, ..., 50}$
			\end{itemize}
			\item \textbf{Continuous random variable}: A random variable defined over a continuous sample space
			\begin{itemize}
				\item Monthly income, $S = {X: 0 \leq x < \infty}$
				\item Monthly profit, $S = {X: -\infty < x < \infty}$
			\end{itemize}
		\end{itemize}
	\end{frame}


	\section{Probability Distribution}

	\begin{frame}{Probability Distribution}
		\begin{itemize}
			\item Probability distribution means the distribution of the probabilities among the different values of a random variable
			\item For example, when tossing a coin twice, the sample space, $S = \{HH, HT, TH, TT\}$
			\item Let an RV, $X = $ number of heads in two coin tosses, then $X$ can take the values: $0, 1, 2$
			\item Then the probability of each value of $X$:
			\begin{center}
				\begin{tabular}{c|c}
					\hline
					$x$ & $P(X = x)$ \\
					\hline
					$0$ & $1/4$ \\
					$1$ & $2/4$ \\
					$2$ & $1/4$ \\
					\hline
				\end{tabular} \label{cointosstable}
			\end{center}
		\end{itemize}
	\end{frame}


	\begin{frame}{Types of Probability Distributions}
		Depending on the type of variable, distributions are of two types:
		\begin{itemize}
			\item \textbf{Discrete probability distribution}: probability distribution of a discrete random variable
			\item \textbf{Continuous probability distribution}: probability distribution of a continuous random variable
		\end{itemize}
	\end{frame}


	\begin{frame}{Discrete probability distribution}
		\begin{itemize}
			\item The probability distribution of a discrete random variable is a table, graph, formula, or other device used to specify all possible values of a discrete random variable along with their respective probabilities
			\item If we let the discrete probability distribution be represented by the function $p(x)$, then $p(x) = P(X = x)$ is the probability of the discrete random variable $X$ to assume a value $x$
			\item $p(x)$ is called a probability mass function (PMF)
		\end{itemize}
	\end{frame}


	\begin{frame}{Probability Mass Function (PMF)}
		A function, p(x), of a discrete random variable X will be called a PMF, if and only if all the following conditions are satisfied:
		\begin{enumerate}
			\item $p(x) \ge 0; \, \forall x$
			\item $\displaystyle \sum_{x} p(x) = 1$
			\item $P(X = a) = p(a)$
		\end{enumerate}
	\end{frame}


	\begin{frame}{Example: PMF}
		In page \ref{cointosstable}, the probability mass function of the variable $X$ representing the number of heads when a coin is tossed twice is given. It is an example of a PMF.

		\begin{center}
				\begin{tabular}{c|c}
					\hline
					$x$ & $P(X = x)$ \\
					\hline
					$0$ & $1/4$ \\
					$1$ & $2/4$ \\
					$2$ & $1/4$ \\
					\hline
				\end{tabular}
			\end{center}
	\end{frame}



    \section*{Thank you.}

\end{document}