\documentclass[12pt, aspectratio=169]{beamer}

\input{../header}
\usepackage{booktabs}
\usepackage[misc]{ifsym}


\title{Summarizing Data}
\author{Md. Aminul Islam Shazid}
\date{}


\begin{document}
    {
		\setbeamertemplate{footline}{}    % NO FOOTLINE FOR THESE TWO FRAMES
		\addtocounter{framenumber}{-2}    % not counting the title page and the outline in frame numbers

		\begin{frame}
			\titlepage
		\end{frame}

		\begin{frame}{Outline}
            \vfill
			\tableofcontents[subsectionstyle=hide]
            \vfill
		\end{frame}
	}

	\section{Introduction}

    \begin{frame}{Summarizing Data}
        \begin{itemize}
            \item Raw data by itself is not useful
            \item Need to extract insight from data
            \item Data needs to be summarized to gain insights
            \item Can summarize in two ways:
            \begin{itemize}
                \item Tabular summary
                \item Graphical summary
            \end{itemize}
        \end{itemize}
    \end{frame}


    \section{Tabular Summary}

    \subsection{Frequency Table}

    \begin{frame}{Frequency Table}
        \begin{itemize}
            \item Organizes data by listing distinct values or classes
            \item Shows how often each value or class occurs
            \item Can be constructed for categorical or numerical data:
             \begin{itemize}
                \item For categorical variable, it shows how many times each category appears in the sample
                \item For discrete numeric variable, it shows how many times each value appears in the sample
                \item For continuous numeric variable, classes or bins are formed first, then the table shows how many values fall in each class
             \end{itemize}
            \item Forms the basis for graphical summaries like bar charts and histograms
        \end{itemize}
    \end{frame}


    \begin{frame}{Example: Frequency Table for Categorical Variable}
        Suppose we have a discreete variable with four categories: A, B, C and D. The data in the sample is: A, B, A, D, A, A, B, C, C, B, A, A, C, D, D, B, D, C, C, B, C. The Frequency table would be:\\
        \begin{center}
            \begin{tabular}{ccc}
                \textbf{Category} & \textbf{Tally} & \textbf{Frequency}\\
                \toprule
                A & \StrokeFive \StrokeOne & 6\\
                B & \StrokeFive & 5\\
                C & \StrokeFive \StrokeOne & 6\\
                D & \StrokeFour & 4\\
                \bottomrule
            \end{tabular}
        \end{center}
        \vspace{0.5em}
        The tally column is only used to keep track of the data when counting by hand.\\[0.5em]
        One can add a percentage column in necessary.
    \end{frame}


    \section*{Questions?}


\end{document}