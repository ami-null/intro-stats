\documentclass[12pt, aspectratio=169]{beamer}

\input{../header}
\usepackage{booktabs}
\usepackage[misc]{ifsym}


\title{Summarizing Data}
\author{Md. Aminul Islam Shazid}
\date{}


\begin{document}
    {
		\setbeamertemplate{footline}{}    % NO FOOTLINE FOR THESE TWO FRAMES
		\addtocounter{framenumber}{-2}    % not counting the title page and the outline in frame numbers

		\begin{frame}
			\titlepage
		\end{frame}

		\begin{frame}{Outline}
            \vfill
			\tableofcontents[subsectionstyle=hide]
            \vfill
		\end{frame}
	}

	\section{Introduction}

    \begin{frame}{Summarizing Data}
        \begin{itemize}
            \item Raw data by itself is not useful
            \item Need to extract insight from data
            \item Data needs to be summarized to gain insights
            \item Can summarize in two ways:
            \begin{itemize}
                \item Tabular summary
                \item Graphical summary
            \end{itemize}
        \end{itemize}
    \end{frame}


    \section{Tabular Summary}

    \subsection{Frequency Table}

    \begin{frame}{Frequency Table}
        \begin{itemize}
            \item Organizes data from the sample by listing distinct values or classes
            \item Shows the frequency of each class in the sample
            \item Can be constructed for categorical or numerical data:
             \begin{itemize}
                \item For categorical variable, it shows the number of observations in each category
                \item For discrete numeric variable, it shows how many times each value has been observed
                \item For continuous numeric variable, classes or bins are formed first, then the table shows how many values fall in each class
             \end{itemize}
            \item Forms the basis for graphical summaries like bar charts and histograms
        \end{itemize}
    \end{frame}


    \begin{frame}{Example: Frequency Table for Categorical Variable}
        Suppose we have a discreete variable with four categories: A, B, C and D. The data in the sample is: A, B, A, D, A, A, B, C, C, B, A, A, C, D, D, B, D, C, C, B, C. The Frequency table would be:\\
        \begin{center}
            \begin{tabular}{ccc}
                \textbf{Category} & \textbf{Tally} & \textbf{Frequency}\\
                \toprule
                A & \StrokeFive \StrokeOne & 6\\
                B & \StrokeFive & 5\\
                C & \StrokeFive \StrokeOne & 6\\
                D & \StrokeFour & 4\\
                \bottomrule
            \end{tabular}
        \end{center}
        \vspace{0.5em}
        The tally column is only used to keep track of the data when counting by hand.\\[0.5em]
        One can add a percentage column in necessary.
    \end{frame}


    \begin{frame}{Frequency Table for continuous Numeric Variable}
        \begin{itemize}
            \item Data is grouped into classes or groups
            \item Each class has the same class interval (the difference between the upper limit and the lower limit of each class)
        \end{itemize}
        \vspace{0.5em}
        Before making the table, need to decide the value of:
        \begin{itemize}
            \item Either the number of classes or groups
            \item Or, the class interval for each class or group
        \end{itemize}
    \end{frame}


    \begin{frame}{Frequency Table for continuous Numeric Variable (cont.)}
        There is no hard and fast rules for setting the classes or the class interval.\\[0.5em]
        
        It usually depends on the variable and its range (difference between the highest value and the lowest value).\\[0.5em]
        \begin{itemize}
            \item For example, if the range is 50, then with class interval of 10, there will be 5 classes
            \item Conversely, if 10 classes are wanted, then the class interval will be 5
        \end{itemize}
        \vspace{0.5em}
        Too many or too few classes may fail to reveal the basic shape of the data.
    \end{frame}


    \begin{frame}{Frequency Table for continuous Numeric Variable (cont.)}
        Sometimes, the classes are determined through subject-matter considerations.\\[0.5em]
        
        For example, in case of vitamin D levels:\\[0.5em]
        \begin{itemize}
            \item Values lower than 10 ng/mL would be classified as deficient
            \item 10 to 30 is insufficient
            \item 30 to 100 is sufficient
            \item Above 100 is considered toxic
        \end{itemize}
    \end{frame}


    \begin{frame}{Example}
        Make a frequency table with the following data: 30, 40, 5, 110, 11, 15, 55, 20, 130, 45, 30, 47, 52, 68, 105, 62, 52, 98, 76, 85, 83, 91, 49, 38, 57, 27, 23, 42, 9, 65\\[0.5em]

        \textbf{Solution}:\\[0.5em]
        The range in the sample is = highest value - lowest value = 130-5 = 125\\[0.5em]
        Therefore, with 5 classes, the class interval would = 125/5=25
    \end{frame}


    \begin{frame}{Example (cont.)}
        \begin{center}
            \begin{tabular}{lcc}
                \textbf{Category} & \textbf{Tally} & \textbf{Frequency}\\
                \toprule
                5-30 & \StrokeFive \StrokeTwo & 7\\
                30-55 & \StrokeFive \StrokeFive & 10\\
                55-80 & \StrokeFive \StrokeOne & 6\\
                80-105 & \StrokeFour & 4\\
                105-130 & \StrokeThree & 3\\
                \bottomrule
            \end{tabular}
        \end{center}
        For each class, the upper limit is excluded and the lower limit is included, except for the last class in which the upper limit is also included. This is called upper limit exclusive and lower limit inclusive.\\[0.5em]
        
        This is necessary because the variable is continuous, so there is no gap between the values of each class.
    \end{frame}


    \subsection{Contingency Table}

    \begin{frame}{Contingency Table}
        \begin{itemize}
            \item A contingency table is used to summarize the relationship between two (or more) categorical variables
            \item Data are arranged in rows and columns, where each cell contains the frequency for a specific combination of categories
            \item Row totals, column totals, and a grand total are often included
            \item It helps identify patterns, associations, or possible dependence between variables
        \end{itemize}
    \end{frame}


    \begin{frame}{Example: Contingency Table}
        \begin{center}
        \begin{tabular}{l|cc|c}
            \toprule
             & \multicolumn{2}{c|}{\textbf{Lung Cancer}} & \\
            \cmidrule(lr){2-3}
            \textbf{Smoking}   & \textbf{Yes} & \textbf{No} & \textbf{Total} \\
            \midrule
            Smoker                 & 45            & 55           & 100 \\
            Non-smoker             & 10            & 90           & 100 \\
            \midrule
            \textbf{Total}         & 55            & 145          & 200 \\
            \bottomrule
        \end{tabular}
        \end{center}
        \vspace{0.5em}
        In the above, it can be seen that lung cancer percentage is higher among the smokers compared to those who do not smoke.
    \end{frame}


    \section*{Questions?}


\end{document}